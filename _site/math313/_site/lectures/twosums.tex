\documentclass{article}
\usepackage[utf8]{inputenc}
\usepackage{amsmath}

\title{Two sums}
\author{Matthew Rudd}
\date{August 17, 2020}

\begin{document}

\maketitle

\section{Sum of the first $n$ integers}

Here's a derivation of the formula for the sum of the first $n$ positive integers, $1, 2, \ldots n$.
Call the sum $S$, and write the expression in two equivalent ways:
$$ 1 + 2 + \ldots + n-1 + n = S \ , $$
$$ n + (n-1) + \ldots + 2 + 1 = S $$
The sum of each column on the left is $(n+1)$ ; since there are $n$ columns, it follows that
$$ n (n+1) = 2S \ . $$
Solving for $S$ yields the formula, namely
$$ \sum_{ i=1 }^{n}{ i } \ = \ \frac{ n(n+1) }{ 2 } \  . $$
This is an example of a direct proof.

\section{Sum of the first $n$ squares}

What about the sum of the first $n$ squares,
$$ 1 + 4 + 9 + \ldots + n^2 = \sum_{i=1}^{n}{ i^2 } \ ? $$
A little experimentation does not seem to reveal the formula, but I claim that
$$ \sum_{i=1}^{n}{ i^2 } = \frac{ n(n+1)(2n+1) }{ 6 } \ .$$
Proving this is a job for mathematical induction! Verifying the base case, $n=1$, is easy, since
$$ \frac{1(2)(3) }{6} = 1 = 1^2 \ . $$
Now suppose that the formula holds for some integer $k$, so that
$$ \sum_{i=1}^{k}{ i^2 } = \frac{ k(k+1)(2k+1) }{ 6 } \ .$$ 
This is the \textit{induction hypothesis}. Using the induction hypothesis, we compute:

\begin{align*}
\sum_{i=1}^{k}{ i^2 } + (k+1)^2 & = \frac{ k(k+1)(2k+1) }{ 6 } + (k+1)^2 \\
& = (k+1) \left( \frac{k(2k+1)}{6} + k+1 \right) \\
& = (k+1) \left( \frac{ 2 k^2 + k + 6k + 6 }{6} \right) \\
& = (k+1) \left( \frac{ 2k^2 + 7k + 6 }{ 6 } \right) \\
& = (k+1) \left( \frac{ (k+2)(2k+3) }{ 6 } \right) \\
& = \frac{ (k+1)(k+2)( 2(k+1) + 1 )}{6} \, .
\end{align*}
The proposed formula when $n=k+1$ therefore follows from the induction hypothesis, so the formula 
holds for all integers $n \geq 1$.

\end{document}
